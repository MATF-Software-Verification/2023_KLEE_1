\documentclass[a4paper]{article}

\usepackage{color}
\usepackage{xurl}
\usepackage[T2A]{fontenc} % enable Cyrillic fonts
\usepackage[utf8]{inputenc} % make weird characters work
\usepackage{csquotes}
\usepackage{graphicx}
\usepackage{subfigure}
\usepackage{float}
\usepackage[english,serbian]{babel}
\usepackage{listings}
\usepackage{amsthm}
\usepackage{amsmath}

\usepackage[unicode]{hyperref}
\hypersetup{colorlinks,citecolor=green,filecolor=green,linkcolor=blue,urlcolor=blue}

\newtheorem{theorem}{Theorem}[section]
\newtheorem{lemma}[theorem]{Lema}

\title{\textit{Beam Stack Search} algoritam pretrage u okviru alata za simboličko izvršavanje KLEE\\ \small{Seminarski rad u okviru kursa\\Verifikacija softvera\\Matematički fakultet}}
\author{Aleksandar Stefanović, 1021/2023 \and Petar Đorđević, 1088/2022}

\begin{document}

\maketitle

\begin{abstract}

\end{abstract}

\tableofcontents

\newpage

\section{Uvod}

Simboličko izvršavanje u svom osnovnom obliku predstavlja tehniku statičke verifikacije programa, tj. verifikacije programa bez njegovog pokretanja, u kojoj se umesto konkretnog stanja programa tokom izvršavanja prati njegovo simboličko stanje \cite{SymExec-King-10.1145/360248.360252}, u cilju otkrivanja grešaka, automatskog generisanja testova sa velikom pokrivenošću koda i slično. Ovo podrazumeva praćenje simboličkih vrednosti promenljivih koje se javljaju u programu, kao i tzv. uslova putanja - logičkih formula nad simboličkim vrednostima tih promenljivih koje moraju da budu zadovoljene da bi izvršavanje stiglo do neke tačke u programu.

Svaka naredba kontrole toka, poput naredbi grananja ili petlji, može prouzrokovati više putanja izvršavanja programa. Ove putanje, zajedno sa pridruženim uslovima putanja i simboličkim vrednostima promenljivih, indukuju simboličko stablo izvršavanja programa. Osim za najtrivijalnije programe, simboličko stablo izvršavanja je nemoguće u potpunosti istražiti. Simboličko stablo izvršavanja program koji sadrži samo $30$ naredbi grananja bi, zanemarujući moguće nedostižne putanje, sadržalo bi $2^{30}$ različitih putanja izvršavanja. Ukoliko program sadrži i petlje, njegovo stablo izvršavanja bi potencijalno bilo i beskonačno veliko (ukoliko i sam uslov prekida petlje ima simboličku vrednost).

Upravo eksplozija broja stanja u simboličkom stablu izvršavanja je jedan od glavnih problema sa kojim se susreću alati za simboličko izvršavanje. Neke od tehnika koje alati koriste da bi umanjili efekat ovog problema su odsecanje nedostižnih putanja, spajanje stanja, zadavanje preduslova i aproksimacija petlji \cite{SurveySymExec-CSUR18}. Međutim, čak i uz primenu ovih tehnika, najčešće nije moguće eliminisati dovoljan broj stanja da bi se efikasno istražilo celokupno simboličko stablo izvršavanja, pa je od velike važnosti izbor stanja, tj. putanja, za ispitivanje.

Izbor narednog stanja za ispitivanje zavisi od primenjene strategije za obilazak puteva. Dve osnovne strategije su \textit{DFS}, tj. pretraga u dubinu, i \textit{BFS}, tj. pretraga u širinu. Iako imaju svoje primene, obe strategije imaju svoje mane - pretraga u dubinu ima tendenciju zaglavljivanja u \enquote{dubokim} putanjama indukovanih petljama ili rekurzijom, dok pretraga u širinu povlači izuzetno veliko memorijsko zauzeće usled potrebe za istovremenim čuvanjem čitavog nivoa čvorova simboličkog stabla izvršavanja. Postoje i hibridne tehnike između ova dva pristupa, poput algoritma \textit{BFS/DFS} \cite{BFS/DFS-StrahinjaStanojevic}. Često se primenjuju i tehnike koje koriste randimozovane varijante algoritama pretrage, gde se stanjima koja imaju neku dobru osobinu može dodeliti prednost prilikom izbora, poput algoritma koji podrazumevano koristi alat za simboličko izvršavanje \verb|KLEE| \cite{KLEE-paper-10.5555/1855741.1855756}.

U nastavku rada će biti predstavljena upotreba \textit{Beam Stack Search} algoritma pretrage, kao i njegova implementacija u okviru alata za simboličko izvršavanje \verb|KLEE|. Na kraju, biće izvršena uporedna analiza performansi ovog algoritma sa nekim od algoritama koje pruža alat \verb|KLEE|.

\section{Opis algoritma pretrage \textit{Beam Stack Search}}

\addcontentsline{toc}{section}{Literatura}
\appendix
\bibliography{literatura}
\bibliographystyle{plain}

\end{document}
